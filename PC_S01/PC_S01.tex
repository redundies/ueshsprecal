\documentclass[14pt,aspectratio=169]{beamer}

\usepackage[utf8]{inputenc}
\usetheme{Warsaw}
\setbeamertemplate{navigation symbols}{}
\setbeamertemplate{headline}{}
\setbeamercovered{transparent}
\mode<presentation>

\begin{document}

%META DATA
\title{Session 1 : Course Introduction}
\subtitle{\textit{Precalculus: A Problem-Solving Approach}}
\author{JING ARQUERO}
\institute{{\normalsize University of the East - Manila} \\ {\normalsize C.M. Recto Ave., Manila, Philippines}}
\date{All Rights Reserved {\textrm{\copyright}} 2019}

\begin{frame}
 \maketitle
\end{frame}

\begin{frame}{Session Outline}
 \begin{enumerate}
  \item Course Overview
  \item Goals \& Skills
  \item Marking System
  \item Requirements
  \item Classroom Policies
  \item References
 \end{enumerate}

\end{frame}



\begin{frame}{Course Overview}
 \begin{block}{Precalculus}
 A higher mathematics bridging program.
  \begin{itemize}
   \item Analytic Geometry
   \item Trigonometry
   \item Mathematical Induction
  \end{itemize}
 \end{block}

 \textbf{REQUISITES}
 \begin{itemize}
  \item Algebraic Fractions \& Rationalization
  \item Factoring Techniques \& Completing the Square
  \item Plotting \& Graphing on the Cartesian Plane
 \end{itemize}


\end{frame}


\begin{frame}{Goals \& Skills}
 \begin{block}{Course Goals}
  \begin{itemize}
   \item Give Meaning to what had been learned
   \item Understand what had been learned
   \item Apply what had been learned in real-life
  \end{itemize}

 \end{block}

 \begin{block}{Skills Development}
  \begin{itemize}
   \item Critical Thinking \& Problem-Solving Skills
   \item Creativity, Flexibility \& Self-Direction
   \item Scientific, Cultural \& Economic Awareness
  \end{itemize}

 \end{block}


\end{frame}


\begin{frame}{Marking System}
 \begin{block}{25\% Written Works}
 \begin{itemize}
  \item Seatwork \& Homework
  \item Laboratory \& Quizzes
 \end{itemize}

 \end{block}

 \begin{block}{45\% Performance Tasks}
 \begin{itemize}
  \item Tasks, Journal \& Notebook
  \item Attendance, Behavior \& Participation
 \end{itemize}
 \end{block}

 \begin{block}{30\% Examinations}
 \begin{itemize}
  \item Prelims, Midterms \& Finals
  \item Remediation, \textit{if applicable}
 \end{itemize}

 \end{block}



\end{frame}


\begin{frame}{Requirements}
 \begin{itemize}
  \item Book, \textit{if available}
  \item Math Notebook, \textit{preparedimmediately}
  \item Ruler, Pencil \& Colored Pens
  \item Portfolio Folder, \textit{Task No. 2}
  \item Physical Scientific Calculator
  \item Graphing Calculator, \textit{preferably GeoGebra}
 \end{itemize}

\end{frame}



\begin{frame}{Classroom Policies}
 \begin{block}{Classroom Officers}
  \begin{itemize}
   \item Offer assistance \& mediate on class affairs
   \item Help maintain classroom discipline
  \end{itemize}

 \end{block}

 \begin{block}{Late Submissions}
  \begin{itemize}
   \item Excuse slips noted by class adviser
   \item Deduction of five points per day of delay
  \end{itemize}

 \end{block}

 \begin{alertblock}{General Policies}
  \begin{itemize}
   \item Consultations occur a week before posting
   \item Total Silence during lecture-discussions
  \end{itemize}

 \end{alertblock}



\end{frame}

\begin{frame}{References}
\textbf{DOCUMENTATION}\\
 This slide presentation is made with {\textrm \LaTeX}.
 The source code is available at:
 \texttt{https://github.com/redundies/ueshsprecal}
\end{frame}


\end{document}
